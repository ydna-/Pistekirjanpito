\documentclass[a4paper,12pt, titlepage]{article}
\usepackage{amssymb,amsthm,amsmath}
\usepackage[finnish]{babel}
\usepackage[T1]{fontenc}
\usepackage[utf8]{inputenc}
\usepackage{graphicx}
\linespread{1.24}
\sloppy

\title{Tietokantasovellus \\ Dokumentaatio}
\author{Andreas Niskanen}
\date{\today}

\begin{document}

\maketitle

\tableofcontents

\newpage

\section{Johdanto}

Kisällikursseilla opiskelijat palauttavat viikoittain harjoitustehtäviä,
joista he voivat ansaita lisäpisteitä kurssisuoritukseen. Tehtäviä on
kahdenlaisia. Tähtitehtävät tarkistetaan, ja niistä saa pisteitä vain
silloin, kun tehtävä on tehty oikein. Tehtäviä saa kuitenkin korjata ja
palauttaa uudelleen. Tähdettömiä tehtäviä ei sen sijaan tarkisteta.\\
Tämä käytäntö vaatii toimiakseen pistekirjanpitojärjestelmän, jonka avulla
ohjaajat voivat kirjata opiskelijan palauttamia tehtäviä ja näiden korjauksia
mahdollisimman sujuvasti. Ohjaajan tulee myös nähdä opiskelijan palauttamat
tehtävät sopivassa muodossa ja muokata näitä tarpeen mukaan.\\
Kurssin opettajan vastuulla on lisätä järjestelmään kursseja ja poistaa
niitä järjestelmästä, sekä lisätä kullekin kurssille harjoitustehtäviä ja
hyväksyä ohjaajia, joilla on oikeudet muokata vain oman kurssin palautuksia.
Lisäksi opettajat tarvitsevat tiettyjä tilastollisia tunnuslukuja (esimerkiksi
palautettujen tehtävien määristä), jotka tulee näkyä kunkin kurssin yhteydessä.\\
Kyseinen pistekirjanpitojärjestelmä toteutetaan tietojenkäsittelytieteen
laitoksen users-palvelimelle käyttäen PostgreSQL-tietokantapalvelinta.
Palvelimen puolella käytetään PHP-ohjelmointikieltä, jonka avulla haetaan
tarvittava data tietokannasta ja tarjotaan REST-rajapinta selaimen puolella
käytettävän AngularJS-ohjelmistokehyksen käytettäväksi.

\newpage

\section{Käyttötapaukset}

\subsection{Käyttötapauskaavio}

\includegraphics[scale=0.5]{kayttotapauskaavio}

\subsection{Käyttäjäryhmät}

\begin{description}
	\item[Opiskelija] \hfill \\
	Opiskelijalla tarkoitetaan ketä tahansa, joka päätyy kyseisen
	järjestelmän web-sivulle. Kaikki muut käyttäjäryhmät kuuluvat
	myös tähän käyttäjäryhmään.
	\item[Ohjaaja] \hfill \\
	Ohjaaja on järjestelmään rekisteröitynyt käyttäjä, jonka vastuulla
	on oman kurssin palautukset ja jonka rekisteröityminen on hyväksytty
	opettajan toimesta.\\
	\item[Opettaja] \hfill \\
	Opettaja on järjestelmään rekisteröitynyt käyttäjä, jonka
	vastuulla on omat kurssit ja jonka rekisteröityminen on hyväksytty
	adminin toimesta.
	\item[Admin] \hfill \\
	Adminilla tarkoitetaan järjestelmän ylintä käyttäjäluokkaa.
\end{description}

\subsection{Käyttötapauskuvaukset}

\begin{description}
	\item[Opiskelijan käyttötapaukset] \hfill \\
	\begin{description}
		\item[Pisteiden tarkastelu] \hfill \\
		Opiskelija pystyy järjestelmän etusivulta tarkastamaan
		omat pisteensä valitsemalla kurssin ja syöttämällä
		oman kurssitunnuksensa.
		\item[Muut käyttötapaukset:] rekisteröityminen, kirjautuminen
	\end{description}
	\item[Ohjaajan käyttötapaukset] \hfill \\
	\begin{description}
		\item[Tehtävien kirjaaminen] \hfill \\
		Ohjaajan tehtävänä on kirjata järjestelmään oman kurssin tehtäviä.
		Ohjaajan tulee myös tarvittaessa pystyä katsomaan ja muokkaamaan
		jo kirjattuja opiskelijan harjoituksen tehtäviä.
		\item[Harjoituksen lisääminen] \hfill \\
		Ohjaajan tehtäviin kuuluu myös lisätä omille kursseille harjoituksia
		määrittelemällä tehtävien lukumäärä. Ohjaaja pystyy tarvittaessa
		myös muokkaamaan ja poistamaan harjoituksia.
		\item[Taulukoiden vienti] \hfill \\
		Ohjaaja pystyy tulostamaan järjestelmästä taulukkona tietyn kurssin
		kaikkien opiskelijoiden palautettujen tehtävien lukumäärät, esimerkiksi
		.csv-muodossa.
		\item[Muut käyttötapaukset:] rekisteröityminen, kirjautuminen
	\end{description}
	\item[Opettajan käyttötapaukset] \hfill \\
	\begin{description}
		\item[Ohjaajan autentikointi] \hfill \\
		Kun ohjaaja rekisteröityy järjestelmään tietyn kurssin ohjaajaksi,
		kurssin vastuuopettajan tulee vahvistaa ohjaajan käyttäjä.
		\item[Kurssin lisääminen] \hfill \\
		Opettajan tehtäviin kuuluu lisätä järjestelmään omia kursseja.
		Opettaja pystyy tarvittaessa myös muokkaamaan ja poistamaan kursseja.
		\item[Opiskelijoiden lisääminen] \hfill \\
		Opettajan tehtäviin kuuluu myös lisätä omille kursseille opiskelijoita
		kurssitunnuksen mukaan. Opettaja pystyy tarvittaessa myös muokkaamaan
		ja poistamaan opiskelijoita.
		\item[Muut käyttötapaukset:] rekisteröityminen, kirjautuminen,
		tehtävien kirjaaminen, harjoituksen lisääminen, taulukoiden vienti
	\end{description}
	\item[Adminin käyttötapaukset] \hfill \\
		\item[Opettajan autentikointi] \hfill \\
		Kun opettaja rekisteröityy järjestelmään, järjestelmän adminin
		tulee vahvistaa opettajan käyttäjä.
		\item[Muut käyttötapaukset:] rekisteröityminen, kirjautuminen,
		tehtävien kirjaaminen, harjoituksen lisääminen, taulukoiden vienti,
		kurssin lisääminen, opiskelijoiden lisääminen
\end{description}

\section{Järjestelmän tietosisältö}

\subsection{Käsitekaavio}

\includegraphics[scale=0.5]{kasitekaavio}

\subsection{Tietokohteet}

\section{Relaatiotietokantakaavio}

\includegraphics[scale=0.5]{relaatiokaavio}

\end{document}
